\usepackage[utf8]{inputenc}
\usepackage[english]{babel}
\usepackage{float}
%\usepackage{courier}
\usepackage{booktabs}
%\usepackage[a4paper, margin=2.2cm]{geometry}

\ifprintunizar
	\usepackage[a4paper, 
				tmargin=45mm, 
				lmargin=40mm, 
				textwidth=130mm, 
				textheight=210mm, 
				headsep=5mm]{geometry}
\else
	\usepackage[a4paper,
				tmargin=2.2cm, 
				lmargin=2.2cm,
				rmargin=2.2cm, 
				bmargin=1.0in]{geometry}
\fi

%\usepackage[letterpaper,margin=2.2cm]{geometry}
%\usepackage[pdftex]{graphicx} %% loaded twice
\usepackage[nodayofweek]{datetime}
\usepackage{environ}
%\usepackage{subfig}
\usepackage{subcaption}
\usepackage[labelfont=bf]{caption}
\usepackage{acronym}
\usepackage[normalem]{ulem}
\usepackage{enumitem}
\usepackage{empheq}
\usepackage{amsmath}
\usepackage{amssymb}
\usepackage{extarrows}
\usepackage[makeroom]{cancel} % cross-out terms in equations
\usepackage{xifthen}
\usepackage{twoopt}
\usepackage{wasysym}
\usepackage{fancyhdr}
\usepackage{siunitx}
\usepackage{ifplatform}
%\usepackage[nottoc]{tocbibind}
%\usepackage{patchcmd}
%\usepackage{parskip}
\usepackage{fix-cm} % mandatory for Package cover 
\usepackage{titlesec} % mandatory for Package cover
%\PassOptionsToPackage{pagestyles}{titlesec}
%\PassOptionsToPackage{clearempty}{titlesec}
%\PassOptionsToPackage{explicit}{titlesec}
\usepackage[T1]{fontenc} % mandatory for Package cover 
\usepackage{adjustbox} % mandatory for alt colour cover
\usepackage[superscript, biblabel, nomove]{cite} % citation package and style -> https://tex.stackexchange.com/questions/79591/superscripts-in-bibliography-with-bibtex#79599
\usepackage[protrusion=true,expansion]{microtype} % install cm-super -> https://tex.stackexchange.com/questions/88368/how-do-i-invoke-cm-super#88374
%\usepackage[activate={true,nocompatibility},final,tracking=true,kerning=true,spacing=true,factor=1100,stretch=10,shrink=10]{microtype}
%\usepackage{lmodern}

\ifx\printabletext\trueValue
	\usepackage[monochrome]{xcolor}
\else
	\usepackage[dvipsnames, table]{xcolor}
	\usepackage{xcolor}
%	\usepackage{color, colortbl}
\fi

\ifdefined\mainmatter
	\usepackage{tocbasic} % mainmatter style book
	\usepackage{xpatch} % mainmatter style book
\fi

% activate morewrites if you need more space to write
\usepackage{morewrites} % https://www.ctan.org/pkg/morewrites
\usepackage[most]{tcolorbox}
\usepackage[none]{hyphenat} % do not split words at the en of line
\usepackage{afterpage}
\usepackage{xfrac}
\usepackage{nicefrac} % diagonal fractions --> https://tex.stackexchange.com/questions/3372/how-do-i-typeset-arbitrary-fractions-like-the-standard-symbol-for-5-%C2%BD
\usepackage{newfloat} % float environment for equation listing
%\usepackage[subfigure]{tocloft}
\usepackage{tocloft}
\usepackage{etoolbox}
\usepackage{everyshi}
\usepackage{atbegshi}
\usepackage{atveryend}
\usepackage{yfonts}
\usepackage{todonotes}
\usepackage{tikz}
\usetikzlibrary{shapes.geometric, calc}
%\usetikzlibrary{calc}
\usepackage[siunitx, RPvoltages]{./packages/circuitikzgit}
\usepackage{anyfontsize}
\usepackage{minted}
\usepackage{array}
\usepackage[column=O, math]{cellspace}
\usepackage{makecell} % https://tex.stackexchange.com/questions/558937/how-to-avoid-broken-horizontal-and-vertical-lines-when-using-multirow-and-multic
\usepackage{multirow}
%\usepackage{bigints}
\usepackage{mathtools}
\usepackage{relsize}
\usepackage{soul}
%\usepackage[integrals]{wasysym}
\usepackage[bottom]{footmisc}
\usepackage{setspace}
\usepackage[percent]{overpic}
\usepackage{xhfill}
\usepackage{eqparbox}
%\usepackage{everypage}
%\usepackage{bophook} % important for top footnote space
%\usepackage[printwatermark]{xwatermark}
%\usepackage{./packages/bibspacing}
%\usepackage[tocindentauto]{tocstyle}
%\usetocstyle{KOMAlike} %the previous line resets it
\usepackage{fix-cm} % Fixes warnings about missing fonts; see http://tex.stackexchange.com/questions/32378/xfrac-siunitx-gives-me-a-font-warning
%\usepackage{bookmark} % faster updated bookmarks
\usepackage{hyperref}
%\usepackage{appendix}
%\usepackage[toc,page]{appendix}
%\usepackage{cleveref}
\usepackage[sort=use]{glossaries}[=v4.46] % always load glossaries AFTER hyperref % https://tex.stackexchange.com/questions/290114/how-to-sort-glossaries-entries-by-alphabetical-order-then-by-order-of-appereance
%\usepackage[sort=use]{glossaries-extra}
%\usepackage{catoptions}
\usepackage{./packages/catoptions}
\usepackage{csquotes} % https://tex.stackexchange.com/questions/120722/how-to-modify-a-bibliography-style-to-surround-titles-in-quotes-in-a-bibliograph
% https://tex.stackexchange.com/questions/64761/add-just-a-little-more-padding-to-my-table/381439

% https://tex.stackexchange.com/questions/64761/add-just-a-little-more-padding-to-my-table/381439
\cellspacetoplimit 4pt

% space after figures
%\setlength{\belowcaptionskip}{-5pt}

\iftwoside

	\typeout{@@@@@@@@@@@@@}
	\typeout{@@@@@@@@@@@@@ SELECTED TWO-SIDE}
	\typeout{@@@@@@@@@@@@@}
	
	\makeatletter
		\@twosidetrue
	\makeatother
\else

	\typeout{@@@@@@@@@@@@@}
	\typeout{@@@@@@@@@@@@@ SELECTED ONE-SIDE}
	\typeout{@@@@@@@@@@@@@}
	
	\makeatletter
		\@twosidefalse
	\makeatother
\fi

\newif\iffilepublications

% set true or false to view the list of publications 
\filepublicationstrue

\iffilepublications

	\makeatletter
		\@ifpackageloaded{multibib}{\relax}{
			\usepackage[resetlabels]{multibib} % multiple bibliography
		}
	\makeatother

\fi

\newif\ifnotwindows

\ifwindows
	% do nothing
	\relax
\else
	\notwindowstrue
\fi

\newcommand{\ConfigLinksBlack}{
	
	\hypersetup{
		colorlinks = true,
		linkcolor=black,
		citecolor=black,
		filecolor=black,
		urlcolor=black,
		runcolor=black,
		menucolor=black,
		linkbordercolor=black,
		citebordercolor=black,
		filebordercolor=black,
		urlbordercolor=black,
		runbordercolor=black,
		menubordercolor=black,
		pdfpagemode=UseOutlines,
		hypertexnames = true,
		pdfencoding = auto, 
		psdextra, 
		bookmarksdepth = 4
		bookmarks=true,
		bookmarksopen=true,    
		bookmarksopenlevel=0,
		bookmarksnumbered=true,
		plainpages=false
	}
}

\newcommand{\ConfigLinksColours}{

	% style color setup: https://tex.stackexchange.com/questions/525261/better-default-colors-for-hyperref-links
	
	\definecolor{blue-hyperref}{rgb}{0,0.2,0.6}
	
	\hypersetup{
		colorlinks = true,
		linkcolor=blue-hyperref,
		citecolor=blue-hyperref,
		filecolor=blue-hyperref,
		urlcolor=blue-hyperref,
		runcolor=blue-hyperref,
		menucolor=blue-hyperref,
		linkbordercolor=blue-hyperref,
		citebordercolor=blue-hyperref,
		filebordercolor=blue-hyperref,
		urlbordercolor=blue-hyperref,
		runbordercolor=blue-hyperref,
		menubordercolor=blue-hyperref,
		pdfpagemode=UseOutlines,
		hypertexnames = true,
		pdfencoding = auto, 
		psdextra, 
		bookmarksdepth = 4
		bookmarks=true,
		bookmarksopen=true,    
		bookmarksopenlevel=0,
		bookmarksnumbered=true,
		plainpages=false
	}
	
%	\hypersetup{
%		colorlinks = true
%		,linkcolor=NavyBlue %BrickRed
%		,citecolor=Green
%		,filecolor=Mulberry
%		,urlcolor=NavyBlue
%		,menucolor=BrickRed
%		,runcolor=Mulberry
%		,linkbordercolor=BrickRed
%		,citebordercolor=Green
%		,filebordercolor=Mulberry
%		,urlbordercolor=NavyBlue
%		,menubordercolor=BrickRed
%		,runbordercolor=Mulberry
%	}
}

\ifx\printabletext\trueValue

	\ConfigLinksBlack

\else
	
	\ifdefined\compileblack
	
		\ConfigLinksBlack
	
	\else
	
		\ConfigLinksColours
	
	\fi
\fi

\ifx\printabletext\trueValue
	\typeout{@@@@@@@@@@@@@}
	\typeout{@@@@@@@@@@@@@ PRINTING BLACK AND WHITE TEXT}{}
	\typeout{@@@@@@@@@@@@@}
\else
	\typeout{@@@@@@@@@@@@@}
	\typeout{@@@@@@@@@@@@@ PRINTING RGB TEXT}{}
	\typeout{@@@@@@@@@@@@@}
\fi

\ifx\printableimages\trueValue
	\typeout{@@@@@@@@@@@@@}
	\typeout{@@@@@@@@@@@@@ PRINTING BLACK AND WHITE IMAGES}{}
	\typeout{@@@@@@@@@@@@@}
\else 
	\typeout{@@@@@@@@@@@@@}
	\typeout{@@@@@@@@@@@@@ PRINTING RGB IMAGES}{}
	\typeout{@@@@@@@@@@@@@}
\fi

% https://tex.stackexchange.com/questions/525261/better-default-colors-for-hyperref-links

%\hypersetup{
%	colorlinks = true,
%	linkcolor={[rgb]{0,0.2,0.6}},
%	citecolor={[rgb]{0,0.6,0.2}},
%	filecolor={[rgb]{0.8,0,0.8}},
%	urlcolor={[rgb]{0.8,0,0.8}},
%	runcolor={[rgb]{0.8,0,0.8}},  % defaults to filecolor but it missing an
%	% expansion for this syntax
%	menucolor={[rgb]{0,0.2,0.6}}, % I never set or use this, so we'll see
%	% it to link color
%	linkbordercolor={[rgb]{0,0.2,0.6}},
%	citebordercolor={[rgb]{0,0.6,0.2}},
%	filebordercolor={[rgb]{0.8,0,0.8}},
%	urlbordercolor={[rgb]{0.8,0,0.8}},
%	runbordercolor={[rgb]{0.8,0,0.8}},
%	menubordercolor={[rgb]{0,0.2,0.6}},
%	hypertexnames=false 
%}

%\hypersetup{
%	colorlinks = true,
%	linkcolor = {[rgb]{0,0.2,0.6}},
%	anchorcolor = {[rgb]{0,0.2,0.6}},
%	citecolor = {[rgb]{0,0.2,0.6}},
%	filecolor = {[rgb]{0,0.2,0.6}},
%	urlcolor = {[rgb]{0,0.2,0.6}},
%	hypertexnames=false
%}

%\hypersetup{
%	colorlinks = true,
%	linkcolor = blue,
%	anchorcolor = blue,
%	citecolor = blue,
%	filecolor = blue,
%	urlcolor = blue,
%	hypertexnames=false
%}

% space in glossary entries, bibspacing
%\setlength{\bibitemsep}{.2\baselineskip plus .05\baselineskip minus .05\baselineskip}

% PDF compression parameters: https://latex.org/forum/viewtopic.php?t=21309
\pdfminorversion=5
\pdfcompresslevel=9
\pdfobjcompresslevel=2
\pdfinclusioncopyfonts=1

\typeout{@@@@@@@@@@@@@ pdfcompresslevel=\the\pdfcompresslevel}
\typeout{@@@@@@@@@@@@@}
\typeout{@@@@@@@@@@@@@ pdfobjcompresslevel=\the\pdfobjcompresslevel}
\typeout{@@@@@@@@@@@@@}
\typeout{@@@@@@@@@@@@@ pdfminorversion=\the\pdfminorversion}
\typeout{@@@@@@@@@@@@@}

\DeclareGraphicsExtensions{.eps,.jpg,.png,.pdf}

\newcommand{\keywords}{physics, resume, summary, report, phd, magnonics, quantum, simulations}

% https://tex.stackexchange.com/questions/48067/pdfinfo-doesnt-appear-to-be-working#48070
\ifpdf
	\makeatletter
		\AtBeginDocument{
			\hypersetup{
				pdftitle={\@title~\expandonce\THETITLE},
				pdfauthor={\@author},
				pdfkeywords={\keywords},
				pdfproducer={\expandonce\THEAUTHOR},
				pdfcreator={\expandonce\THEAUTHOR},
				pdfsubject={\expandonce\THETITLE},
				pdfinfo={
					Title={\THETITLE},
					Author={\THEAUTHOR},
					Creator={\THEAUTHOR},
					Producer={\THEAUTHOR}
				}
			}
		}
	\makeatother
\fi

\setlist[itemize]{itemsep=0.005cm}

\newif\ifformat
\formattrue

% Command to avoid style expand in TOC/LOT/LOF
\newcommand{\donotexpandstyles}[1]{
	\AtBeginDocument{\addtocontents{#1}{\protect\formatfalse}}
	\AtEndDocument{\addtocontents{#1}{\protect\formattrue}}
}

% Do not expand styles in table of contents
\donotexpandstyles{toc}
% Do not expand styles in list of tables
\donotexpandstyles{lot}
% Do not expand styles in list of figures
\donotexpandstyles{lof}
% Do not expand styles in list of equations
\donotexpandstyles{equ}

% Do not show styles on TOC/LOT/LOF, put \maybe before styled text
\makeatletter
\DeclareRobustCommand{\maybe}[1]{%
	\ifformat
		\expandafter#1%
	\else
		\expandafter\@firstofone
	\fi
}
\makeatother

% remove word Chapter at the beginning of each chapter
%\titleformat{\chapter}[block]{\normalfont\huge\bfseries}{\thechapter.}{1em}{\Huge}\titlespacing*{\chapter}{0pt}{-19pt}{0pt}
% https://tex.stackexchange.com/questions/62516/how-to-suppress-chapter-in-chapter-while-keeping-numbering
\makeatletter
\def\@makechapterhead#1{%
	\vspace*{-19\p@}%
	{\parindent \z@ \raggedright \normalfont
		\ifnum \c@secnumdepth >\m@ne
			\if@mainmatter
				%\huge\bfseries \@chapapp\space \thechapter
				\Huge\bfseries \thechapter.\space%
				%\par\nobreak
				%\vskip 20\p@
			\fi
		\fi
		\interlinepenalty\@M
		\Huge \bfseries #1\par\nobreak
		\vskip 10\p@
}}
\makeatother

\ifx\printableimages\trueValue
%	\immediate\write18{chmod +x utils/convert-grayscale.sh && ./utils/convert-grayscale.sh &>/dev/null}
	\graphicspath{{./images-grayscale/}}
\else
	\graphicspath{{./images/}}
\fi

% Continous footnote counter
% https://tex.stackexchange.com/questions/10448/continuous-footnote-numbering
\counterwithout{footnote}{chapter}

%% INDENTATION REVIEW
\newif\ifindentfirst
\newif\ifindentall

\indentfirstfalse
\indentallfalse

\def\indentspace{\hspace*{1em}}
\def\noindentspace{\hspace*{0em}}

\ifindentfirst

	% START INDENT ONLY FIRST PARAGRAPH OTHER PARAGRAPHS NOT INDENT

	\makeatletter
	\def\@afterheading{%
		\@nobreaktrue
		\everypar{%
			\if@nobreak
				\@nobreakfalse
				\clubpenalty \@M
				\indentspace%
			\else
				\clubpenalty \@clubpenalty
				\everypar{}%
			\fi}}
	\makeatother
	\setlength{\parindent}{0pt}

	% END INDENT ONLY FIRST PARAGRAPH OTHER PARAGRAPHS NOT INDENT

\else

	\ifindentall
	
		\setlength{\parindent}{1em}
		
	\else
	
		%% START NOT INDENT FIRST PARAGRAPH OTHER PARAGRAPHS INDENT
		
		\makeatletter
		\def\@afterheading{%
			\@nobreaktrue
			\everypar{%
				\if@nobreak
					\@nobreakfalse
					\clubpenalty \@M
					\noindentspace%
				\else
					\clubpenalty \@M
					\indentspace
				\fi}}
		\makeatother
		\setlength{\parindent}{0em}
		
		%% END NOT INDENT FIRST PARAGRAPH OTHER PARAGRAPHS INDENT
	
		%% PATCH INDENT AFTER ENVIRONMENT LIST BASED
		% https://tex.stackexchange.com/questions/112404/reliable-code-for-automatic-noindent-after-specific-environments
		% https://tex.stackexchange.com/questions/23062/an-unexpected-value-of-clubpenalty
		
		\makeatletter
		\let\nearly@afterheading\@afterheading
		\patchcmd\nearly@afterheading
			{\@M} % original temporary setting for \clubpenalty replaced by ...
			{\@clubpenalty\protect\indentspace} % ... or whichever value you deem right
			{}{}
	
		\newcommand*\IndentAfterEnv[1]{%
			\AfterEndEnvironment{#1}{\par\@afterindentfalse\nearly@afterheading}}
		
		\newcommand*\SpaceBeforeEnv[1]{%
			\BeforeBeginEnvironment{#1}{\par\protect\vspace*{0.5em}}}
		\makeatother

		% indent after environment
		\IndentAfterEnv{itemize}
		\IndentAfterEnv{enumerate}
		
		% mandatory for a \parskip of 0.7em
		\SpaceBeforeEnv{itemize}
		\SpaceBeforeEnv{enumerate}
		
		%% END PATCH INDENT AFTER ENVIRONMENT LIST BASED
	\fi

\fi
%% END INDENTATION REVIEW

%% START TOC STYLE
% https://tex.stackexchange.com/questions/154646/is-there-an-easy-way-to-get-the-frontmatter-mainmatter-and-backmatter-in-a-l/154666
% https://tex.stackexchange.com/questions/419555/remove-bold-style-for-chapter-names-in-table-of-contents
% https://tex.stackexchange.com/questions/123774/remove-the-dots-next-to-chapter-in-memoir
% https://tex.stackexchange.com/questions/299074/removing-number-on-toc-but-maintaining-the-position-of-the-text
% http://www.texfaq.org/FAQ-isdef

\ifdefined\mainmatter

	% https://tex.stackexchange.com/questions/439926/different-style-for-backmatter-chapters-in-toc
	% \mainmatter defined
	\typeout{@@@@@@@@@@@@@}
	\typeout{@@@@@@@@@@@@@ mainmatter defined}
	\typeout{@@@@@@@@@@@@@}
	
		% https://tex.stackexchange.com/a/511461
	% https://tex.stackexchange.com/questions/544569/koma-toc-constant-distance-between-section-title-and-pagenumber
	
%	\DeclareTOCStyleEntry[%
%		entryformat=\chapterentryformat,%
%		pagenumberformat=\chapterentryformat,%
%		pagenumberbox=\pagenumberboxsimple,%
%		linefill=\dotfillcustomentryformat,%
%		indent=0pt
%	]{tocline}{chapter}%
	
%	\DeclareTOCStyleEntries[%
%		entryformat=\sectionentryformat,%
%		pagenumberformat=\sectionentryformat,%
%		pagenumberbox=\pagenumberboxsimple,%
%		linefill=\dotfillcustomentryformat,%
%		indent=0pt
%	]{tocline}{section}%
	
%	\DeclareTOCStyleEntry[%
%		entryformat=\subsectionentryformat,%
%		pagenumberformat=\subsectionentryformat,%
%		pagenumberbox=\pagenumberboxsimple,%
%		%linefill=\dotfill,%
%		%indent=0pt
%	]{tocline}{subsection}%
%	
%	\DeclareTOCStyleEntry[%
%		pagenumberbox={\mbox}%
%	]{tocline}{figure}%
%	
%	\newif\ifmainmatterintoc
%	
%	%	\newcommand*\pagenumberbox[1]{\mbox{\hspace{1em}#1}}
%	\newcommand*\pagenumberboxsimple[1]{\mbox{#1}\relax}%
%	
%	%	\newcommand*\dotfillcustomentryformat[1]{{\ifmainmatterintoc\dotfill\else\relax\fi#1}}%
%	
%	\newcommand*\subsectionentryformat[1]{{\ifmainmatterintoc\relax\else\normalfont\itshape\fi#1}}%
%	
%	\newcommand*\sectionentryformat[1]{{\ifmainmatterintoc\normalfont\else\normalfont\itshape\fi#1}}%
%
%	\newcommand*\chapterentryformat[1]{{\ifmainmatterintoc\normalfont\bfseries\else\normalfont\itshape\fi#1}}%
%	
%	\xapptocmd\mainmatter{\addtocontents{toc}{\protect\mainmatterintoctrue}}{}{}%
%	\xapptocmd\backmatter{\addtocontents{toc}{\protect\mainmatterintocfalse}}{}{}%

\else

	% \mainmatter not defined
	\typeout{@@@@@@@@@@@@@}
	\typeout{@@@@@@@@@@@@@ mainmatter NOT defined}
	\makeatletter
	\typeout{@@@@@@@@@@@@@}
	
	\newcommand*{\enableboldchapterintoc}{%
		
		\addtocontents{toc}{\protect\unexpanded{\protect\unexpanded{\string\renewcommand{\protect\cftchapdotsep}{\protect\cftnodots}}}} % remove dots
		\addtocontents{toc}{\string\renewcommand{\protect\cftchapfont}{\protect\normalfont\protect\bfseries}}
		\addtocontents{toc}{\string\renewcommand{\protect\cftchappagefont}{\protect\normalfont\protect\bfseries}}
		\addtocontents{toc}{\string\renewcommand{\protect\cftchapleader}{\protect\bfseries\hfill\relax}}
		%	\addtocontents{toc}{\string\renewcommand{\protect\cftchapleader}{\protect\bfseries\protect\cftdotfill{\protect\cftsecdotsep}}} % dot leaders in bold
	}
	
	\newcommand*{\enableitalicchapterintoc}{%
		
		\addtocontents{toc}{\protect\unexpanded{\protect\unexpanded{\string\renewcommand{\protect\cftchapdotsep}{\protect\cftnodots}}}} % remove dots
		\addtocontents{toc}{\string\renewcommand{\protect\cftchapfont}{\protect\normalfont\protect\itshape}}
		\addtocontents{toc}{\string\renewcommand{\protect\cftchappagefont}{\protect\normalfont\protect\itshape}}
		\addtocontents{toc}{\string\renewcommand{\protect\cftchapleader}{\protect\itshape\hfill\relax}}
	}
	
	\newcommand*{\disablestylechapterintoc}{%
		
		\addtocontents{toc}{\string\renewcommand{\protect\cftchappagefont}{\protect\normalfont}}
		\addtocontents{toc}{\string\renewcommand{\protect\cftchapfont}{\protect\normalfont}}
		\addtocontents{toc}{\string\renewcommand{\protect\cftchapleader}{\protect\normalfont\protect\cftdotfill{\protect\cftsecdotsep}}}% 
	}
	
	\newcommand{\frontmatter}{%
		
		\disablestylechapterintoc
		\enableitalicchapterintoc
		
		\cleardoublepage
		\pagenumbering{Roman}
	}
	
	\newcommand{\mainmatter}{%
		
		\disablestylechapterintoc
		\enableboldchapterintoc
		
		\cleardoublepage
		\pagenumbering{arabic}
	}
	
	\newcommand{\backmatter}{%
		
		\disablestylechapterintoc
		\enableitalicchapterintoc
		
		\if@openright
			\cleardoublepage
		\else
			\clearpage
		\fi
	}
	
	\makeatother

\fi


\ifdefined\mainmatter
	% https://tex.stackexchange.com/questions/159998/changing-the-dot-size-in-the-table-of-contents
	\makeatletter          % to set the marginpar on the left side in showframe
	\let\Gm@vrules@mpi\Gm@vrules@mpii
	\makeatother
	
	%	\newcommand\mydot[1]{\scalebox{#1}{.}}
	%	\renewcommand\cftdot{\mydot{2}}
	
	%	\setcounter{tocdepth}{2}% only parts and chapters in ToC
	
	%	\makeatletter
	%	
	%	\def\@dotsep{0.072}
	%	\newcommand*\skillmon@pagenumberbox[1]{\makebox{#1}}
	%	\newcommand*\skillmon@numsep{0pt}
	
	%	\DeclareTOCStyleEntry[%
	%			,dynnumwidth=true%
	%			,numwidth=0pt%
	%			,numsep=\skillmon@numsep%
	%			,pagenumberbox=\skillmon@pagenumberbox%
	%			]{tocline}{figures}
	
	\DeclareTOCStyleEntries[%
	%		afterskip = -0.65em plus 0pt, %
	%		numsep=\skillmon@numsep,%
			entryformat=\chapterentryformat,%
			pagenumberformat=\chapterentryformat,%
			pagenumberbox=\pagenumberboxsimple,%
	%		numwidth=0pt,%
	%		pagenumberbox=\skillmon@pagenumberbox,%
	%		linefill=\dotfillcustomentryformat,%
	%		indent=0pt
	%		dynnumwidth=true%
	]{tocline}{chapter}%
	
	%	\DeclareTOCStyleEntry[%
	%%		dynnumwidth=true,%
	%%		numsep=\skillmon@numsep,%
	%%		entryformat=\sectionentryformat,%
	%%		pagenumberformat=\sectionentryformat,%
	%%		pagenumberbox=\pagenumberboxsimple,%
	%%		pagenumberbox=\skillmon@pagenumberbox,%
	%%		linefill=\dotfillcustomentryformat,%
	%		indent=0cm
	%	]{tocline}{figure}%
	
	% https://tex.stackexchange.com/questions/347807/bold-chapter-with-dottedtocline-and-tocbasic
	%	\DeclareTOCStyleEntry[beforeskip=1pt]{tocline}{chapter}
	
	%	https://tex.stackexchange.com/questions/16422/indentation-in-toc-and-lists-of-figures-tables
	\setlength{\cftfigindent}{0pt}
	
	% 	https://tex.stackexchange.com/questions/16422/indentation-in-toc-and-lists-of-figures-tables
	\makeatletter
	\renewcommand\@dotsep{2}% default is 4.5
	\renewcommand*\l@figure{\@dottedtocline{1}{0em}{2.3em}}% Default: 1.5em/2.3em
	\let\l@table\l@figure
	\makeatother
	
	\newif\ifembackmattertoc
	
	\embackmattertoctrue
	
	\ifembackmattertoc
	\relax%
		\DeclareTOCStyleEntry[%
%			raggedentrytext=true,%
%			beforeskip=0pt plus .2pt,
%			beforeskip=0em,%
			dynnumwidth=true,%
%			dynindent=true,
%			numsep=\skillmon@numsep,%
%			onstarthigherlevel=\LastTOCLevelWasSame,
			entryformat=\sectionentryformat,%
			pagenumberformat=\sectionentryformat,%
			pagenumberbox=\pagenumberboxsimple,%
%			pagenumberbox=\skillmon@pagenumberbox,%
%			linefill=\dotfillcustomentryformat,%
%			indent=1pt
		]{tocline}{section}%
		%	
		\DeclareTOCStyleEntry[%
%			raggedentrytext=true,%
%			raggedentrytext,
%			beforeskip=0pt plus .2pt,
%			numwidth=0pt,%
%			onstarthigherlevel=\LastTOCLevelWasSame,
			dynnumwidth=true,%
%			beforeskip=0em,%
%			dynindent=true,
			entryformat=\subsectionentryformat,%
			pagenumberformat=\subsectionentryformat,%
			pagenumberbox=\pagenumberboxsimple,%
%			pagenumberbox=\skillmon@pagenumberbox,%
%			pagenumberbox={\mbox}%
%			linefill=\dotfill,%
%			indent=1pt
		]{tocline}{subsection}%
	
	\fi
	
	%	\DeclareTOCStyleEntry[%
	%		dynnumwidth=true,%
	%%		pagenumberbox=\skillmon@pagenumberbox,%
	%	]{tocline}{figure}%
	
	%	\DeclareTOCStyleEntry[%
	%		dynnumwidth=true,%
	%		pagenumberbox=\skillmon@pagenumberbox,%
	%	]{tocline}{equation}%
	%	\makeatother
	
	%\DeclareTOCStyleEntries[
	%entryformat=\em
	%]{tocline}{subsection}
	
	%	\DeclareTOCStyleEntry[%
	%		pagenumberbox={\mbox}%
	%	]{tocline}{figure}%
	%	
	%	\makeatother
	
	\renewcommand\cftdotsep{2}%
	\renewcommand{\cftsecdotsep}{2}%
	\renewcommand{\cftsubsecdotsep}{2}%
	
	\newif\ifmainmatterintoc
	
	%	\newcommand*\pagenumberbox[1]{\mbox{\hspace{1em}#1}}
	\newcommand*\pagenumberboxsimple[1]{\mbox{#1}\relax}%
	
	%	\newcommand*\dotfillcustomentryformat[1]{{\ifmainmatterintoc\dotfill\else\relax\fi#1}}%
	
	\newcommand*\subsectionentryformat[1]{\ifmainmatterintoc\relax#1\else\normalfont\itshape#1\fi}%
	
	\newcommand*\sectionentryformat[1]{\ifmainmatterintoc\normalfont#1\else\normalfont\itshape#1\fi}%
	
	\newcommand*\chapterentryformat[1]{\ifmainmatterintoc\normalfont\bfseries#1\else\normalfont\itshape#1\fi}%
	
	\xapptocmd\mainmatter{\addtocontents{toc}{\protect\mainmatterintoctrue}}{\relax}{}%
	\xapptocmd\backmatter{\addtocontents{toc}{\protect\mainmatterintocfalse}}{\relax}{}%
	
	%	\renewcommand*{\tableofcontents}{\listoftoc[\contentsname]{toc}}
	%	\DeclareTOCStyleEntries[raggedentrytext]{tocline}{section,subsection,subsubsection,paragraph,subparagraph}
	
	\ifembackmattertoc
		% FIX TOC BREAK PAGE
		% https://tex.stackexchange.com/questions/211642/allow-page-breaks-in-the-toc-after-parts
		\makeatletter%
		\let\old@l@subsection\l@subsection%
		\renewcommand{\l@subsection}[2]{%
			\old@l@subsection{#1}{#2}% Regular setting of \part in ToC
			\pagebreak[0]%
		}% Add a zero penalty (for possible page break)
		\makeatother%
	\fi
\fi

%% END TOC STYLE

% https://tex.stackexchange.com/questions/323455/bold-math-in-title-but-not-in-toc
\newcommand\AvoidBoldMathInTocFor[1]{
	\titleformat*{#1}{\bfseries\boldmath}
}

\AvoidBoldMathInTocFor{\section}
\AvoidBoldMathInTocFor{\subsection}
\AvoidBoldMathInTocFor{\subsubsection}

% Check twoside or oneside document
\makeatletter
% https://tex.stackexchange.com/questions/360785/how-do-i-check-if-a-document-is-oneside-or-twoside#360791
\if@twoside%

	\typeout{@@@@@@@@@@@@@}
	\typeout{@@@@@@@@@@@@@ TWO-SIDE DOCUMENT}
	\typeout{@@@@@@@@@@@@@}
	
	% CLEAN DOUBLE PAGES ON TWOSIDE MODE
	% https://tex.stackexchange.com/questions/62572/twoside-introduces-incorrect-linespacing-at-end-of-section
	\raggedbottom
	% https://stackoverflow.com/questions/491904/how-do-i-remove-blank-pages-coming-between-two-chapters-in-appendix
	\let\cleardoublepage\clearpage
	% END CLEAN DOUBLE PAGES ON TWOSIDE MODE
	
\else%

	\typeout{@@@@@@@@@@@@@}
	\typeout{@@@@@@@@@@@@@ ONE-SIDE DOCUMENT}
	\typeout{@@@@@@@@@@@@@}
	
\fi%  

\makeatother

% brakets for citation superscripts
% https://tex.stackexchange.com/questions/79591/superscripts-in-bibliography-with-bibtex#79599
\makeatletter 
	\renewcommand{\@citess}[1]{\textsuperscript{\,[#1]}} 
\makeatother

% https://tex.stackexchange.com/questions/29084/biblatex-separator-for-citing-multiple-sources-some-of-them-with-page-some-no
%\renewcommand\multicitedelim{\addcomma\space}

% cell spaces global
% https://tex.stackexchange.com/questions/378268/makegapedcells-disables-rowcolor#378630
\setlength\cellspacetoplimit{4pt}
\setlength\cellspacebottomlimit{4pt}

\ifx\printabletext\trueValue
	\relax
\else
	% change colour of glossary links
	% https://tex.stackexchange.com/questions/38544/glossary-links-color
	\def\glossarylinknewcolour{orange}
	
	\ifdefined\compileblack
		\def\glossarylinknewcolour{black}
	\else
		\def\glossarylinknewcolour{orange}
	\fi

	\makeatletter
	
		\newcommand{\glsplainhyperlink}[2]{\colorlet{currenttext}{\glossarylinknewcolour}\colorlet{currentlink}{\@linkcolor}\hypersetup{linkcolor=currenttext}\hyperlink{#1}{#2}\hypersetup{linkcolor=currentlink}}
		\let\@glslink\glsplainhyperlink
		
	\makeatother
\fi

% new spacing between cahpters, sections and subsections
% https://tex.stackexchange.com/questions/53338/reducing-spacing-after-headings
\titlespacing\section{0pt}{8pt plus 4pt minus 2pt}{0pt plus 2pt minus 2pt}%
\titlespacing\subsection{0pt}{8pt plus 4pt minus 2pt}{0pt plus 2pt minus 2pt}%
\titlespacing\subsubsection{0pt}{8pt plus 4pt minus 2pt}{0pt plus 2pt minus 2pt}%